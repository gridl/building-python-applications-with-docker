\documentclass{beamer}

\usepackage[utf8]{inputenc}
\usepackage{listings}

\title{Building Python Applications with Docker}
\subtitle{https://github.com/moshez/building-python-applications-with-docker}
\author{Moshe Zadka -- https://cobordism.com}
\date{PyTexas 2017}
 
\begin{document}
 
\frame{\titlepage}

\begin{frame}
\frametitle{Introduction}
\begin{itemize}
\item Clarify terms \pause
\item Common mistakes \pause
\item Best practices
\end{itemize}
\end{frame}

\begin{frame}[fragile]
\frametitle{What are containers?}
Share-less processes
\end{frame}

\begin{frame}[fragile]
\frametitle{What is Docker?}
\begin{itemize}
\item client \pause
\item server (containerd) \pause
\item runner (runc) \pause
\item hub
\end{itemize}
\end{frame}

\begin{frame}[fragile]
\frametitle{What is a Container image?}
\begin{itemize}
\item Tarball... \pause
\item that looks like a linux root \pause
\item Will also specify some metadata (e.g., entrypoint)
\end{itemize}
\end{frame}

\begin{frame}[fragile]
\frametitle{What is a Dockerfile?}
Builds a Docker image
\begin{lstlisting}
FROM ....
COPY from-context
RUN some command
...
\end{lstlisting}
\end{frame}

\begin{frame}[fragile]
\frametitle{Python application}
\verb|catx/app.py|
\lstinputlisting[language=Python,firstline=3]{catx/app.py}
\verb|catx/__main__.py|
\lstinputlisting[language=Python,firstline=5]{catx/__main__.py}
\end{frame}

\begin{frame}[fragile]
\frametitle{What's bad?}
\verb|bad.docker|
\lstinputlisting{bad.docker}
\end{frame}

\begin{frame}[fragile]
\frametitle{What's wrong?}
\begin{itemize}
\item latest \pause
\item Layer explosion \pause
\item Random system packages \pause
\item Random Python packages \pause
\item PYTHONPATH \pause
\item Shipping build environment \pause
\item Using reference WSGI server
\end{itemize}
\end{frame}

\begin{frame}
\frametitle{Base images}
Choose foundation wisely (and stabley)
\end{frame}

\begin{frame}
\frametitle{Fork tagging}
Renaming images for posterity
\end{frame}

\begin{frame}[fragile]
\frametitle{Fork tagging}
\lstinputlisting[language=Python,firstline=7]{fork-tag.py}
\end{frame}

\begin{frame}[fragile]
\frametitle{Python Artifacts}
\framesubtitle{sdist}
\begin{itemize}
\item What \verb|python setup.py sdist| creates
\item Basically just a tarball of sources
\end{itemize}
\end{frame}

\begin{frame}[fragile]
\frametitle{Python Artifacts}
\framesubtitle{wheel}
\begin{itemize}
\item Fully built and ready
\item Reliable installation
\end{itemize}
\end{frame}

\begin{frame}[fragile]
\frametitle{Python Artifacts}
\framesubtitle{pex}
\begin{itemize}
\item Python executable
\item Installation is "copy"
\item Does not work well with PyPy
\end{itemize}
\end{frame}

\begin{frame}[fragile]
\frametitle{Python Artifacts}
\framesubtitle{Virtual environments}
\begin{itemize}
\item Location specific
\item Can be used with \verb|dh_virtualenv|
\end{itemize}
\end{frame}

\begin{frame}[fragile]
\frametitle{What is Multistage Build?}
\begin{itemize}
\item Build images sequentially \pause
\item Throw away all except last \pause
\item Copy previous image
\end{itemize}
\end{frame}

\begin{frame}[fragile]
\frametitle{Multistage Build}
\begin{lstlisting}
FROM source as name1
...
FROM other-source
...
COPY --from=name1 ...
...
ENTRYPOINT [...]
\end{lstlisting}
\end{frame}

\begin{frame}[fragile]
\frametitle{Better Version 1}
\lstinputlisting[lastline=2]{msb.docker}\pause
\lstinputlisting[firstline=3,lastline=4]{msb.docker}\pause
\lstinputlisting[firstline=5,lastline=7]{msb.docker}\pause
\lstinputlisting[firstline=8,lastline=8]{msb.docker}\pause
\lstinputlisting[firstline=9]{msb.docker}
\end{frame}

\begin{frame}[fragile]
\frametitle{Locking}
\begin{lstlisting}
$ git checkout -b updating-third-party
$ docker build -t temp-image -f lock.docker .
$ docker run --rm -it temp-image > \
  requirements.locked.txt
$ git commit -a -m 'Update 3rd party'
$ git push
$ # Follow code workflow
\end{lstlisting}
\end{frame}

\begin{frame}[fragile]
\frametitle{Locking}
\begin{lstlisting}
FROM moshez/pypy3:2017-10-30T09-29-03-882199
COPY lock-requirements requirements.loose.txt /
ENTRYPOINT ["/lock-requirements"]
\end{lstlisting}
\end{frame}

\begin{frame}[fragile]
\frametitle{Locking}
\begin{lstlisting}
cmd(sys.executable, "-m", "venv", "/envs/loose")
cmd("/envs/loose/bin/pip", "install", "-r",
    "/requirements.loose.txt"])
frozen = subprocess.check_output(
            ["/envs/loose/bin/pip",
             "freeze", "--all"])
with open("/requirements.locked.txt", 'wb') as fp:
    fp.write(frozen)
\end{lstlisting}
\end{frame}

\begin{frame}
\frametitle{Summary}
\begin{itemize}
\item Use multi-stage builds \pause
\item Lock Python requirements \pause
\item Think about your entrypoint
\end{itemize}
\end{frame}

\end{document}
